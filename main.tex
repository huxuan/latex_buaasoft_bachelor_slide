% !Mode:: "TeX:UTF-8"
% +-----------------------------------------------------------------------------
% | File: main
% | Author: huxuan
% | E-mail: i(at)huxuan.org
% | Created: 2012-06-07
% | Last modified: 2012-06-07
% | Description:
% |     An guide example for writing the slide
% |
% | Copyrgiht (c) 2012 by huxuan. All rights reserved.
% | License GPLv3
% +-----------------------------------------------------------------------------
\documentclass[17pt]{beamer}
\usepackage[adobefonts]{ctex}
\usepackage{graphicx}
\usepackage{tabularx}
\usepackage{fix-cm}
\usepackage{color}

\definecolor{title}{RGB}{128,128,128}

\setlength{\paperwidth}{25.4cm}
\setlength{\paperheight}{19.05cm}
\setlength{\textwidth}{.95\paperwidth}
\setlength{\textheight}{\paperheight}

\newcommand{\zihaopt}[2]{\fontsize{#1}{\baselineskip}\selectfont#2}

\setCJKmainfont{SimSun}
\setsansfont{Times New Roman}

\setCJKfamilyfont{xinwei}{STXinwei}
\newcommand{\xinwei}[1]{{\CJKfamily{xinwei}#1}}
\setCJKfamilyfont{hei}{SimHei}
\newcommand{\hei}[1]{{\CJKfamily{hei}#1}}

\setbeamertemplate{navigation symbols}{}

\newcommand{\tutor}[1]{\setbeamertemplate{tutor}{#1}}
\newcommand{\sid}[1]{\setbeamertemplate{sid}{#1}}

% 前言部分唯一需要修改的地方
% 全局变量,对应第一页的内容
\title{牛逼的毕设题目}
\sid{牛逼的学号}
\author{牛逼的你}
\tutor{牛逼的导师}
\institute{牛逼的实习单位}

\setbeamertemplate{title page}{
    \vspace{1cm}

    {
        \setlength{\extrarowheight}{2cm}
        \begin{tabularx}{\textwidth}{@{\hspace{1.75cm}}X}
            \textcolor{title}{\xinwei{\zihaopt{44}{本科生毕设设计答辩}}} \\
            \bf{\textcolor{blue}{\zihaopt{28}{\inserttitle}}} \\
        \end{tabularx}
    }

    \vspace{1.98cm}

    \hei{\zihaopt{31}{
        \setlength{\extrarowheight}{0.3cm}
        \begin{tabularx}{\textwidth}{@{\hspace{3.2cm}}ccc}
            \makebox[6cm]{学\hfill 号} & : & \makebox[7cm]{\hfill \usebeamertemplate{sid} \hfill} \\
            \makebox[6cm]{姓\hfill 名} & : & \makebox[7cm]{\hfill \insertauthor \hfill}\\
            \makebox[6cm]{指\hfill 导\hfill 教\hfill 师}  & :& \makebox[7cm]{\hfill \usebeamertemplate{tutor} \hfill} \\
            \makebox[6cm]{实\hfill 习\hfill 单\hfill 位}  & :& \makebox[7cm]{\hfill \insertinstitute \hfill} \\
        \end{tabularx}
    }}
}

\setbeamertemplate{itemize items}[square]
\setbeamertemplate{itemize subitem}[circ]
\setbeamerfont{itemize/enumerate body}{size*={28}{\baselineskip}}
\setbeamerfont{itemize/enumerate subbody}{size*={24}{\baselineskip}}

\setbeamertemplate{section in toc}[square]
\setbeamercolor{section in toc}{fg=black}
\setbeamerfont{section in toc}{size*={28}{\baselineskip},series=\bfseries}

\setbeamertemplate{frametitle}{\vspace{2cm}\hspace{0.8cm}\insertframetitle}
\setbeamercolor{frametitle}{fg=blue}
\setbeamerfont{frametitle}{size*={44}{\baselineskip},series=\bfseries}

\setbeamertemplate{bibliography item}{\insertbiblabel}
\setbeamercolor{bibliography item}{fg=black}
\setbeamercolor{bibliography entry author}{fg=black}


\begin{document}
\CTEXnoindent

    \usebackgroundtemplate{
        \includegraphics[width=\paperwidth,height=\paperheight]{pic/bg1.png}
    } % 第一页背景 

    \frame{\titlepage}

    \usebackgroundtemplate{
        \includegraphics[width=\paperwidth,height=\paperheight]{pic/bg2.png}
    } % 其它页背景

    % 提纲页,用目录方式生成
    % 这一页是自动生成的,不需要在这里手动写
    \begin{frame}
        \frametitle{汇报内容}
        \tableofcontents
    \end{frame}

    % 正文部分需要修改部分的开始,Slide的主要内容

    \section{目录和页面示例} % Section名称即对应提纲页中的目录项

    \begin{frame}
        \frametitle{这样就是一个页面了,只有标题}
    \end{frame}

    \begin{frame}
        \frametitle{目录和页面不是一对一关系}
    \end{frame}

    \begin{frame}[c]
        \frametitle{垂直居中的页面}
        \begin{itemize}
            \item 我真的是垂直居中的
            \item 默认值就是我
        \end{itemize}
    \end{frame}

    \begin{frame}[t]
        \frametitle{向上对齐的页面}
        \begin{itemize}
            \item 我真的是向上对齐的
            \item 而且你也可以把这两个标签用在后面介绍的分栏中
        \end{itemize}
    \end{frame}

    \section{简单页面示例}

    \begin{frame}
        \frametitle{简单列表示例} 

        \begin{itemize}
            \item 这就是一个列表
            \item 和列表中的一项
            \item[] 这样就可以去掉前面的标记
            \item
            \item 列表的项可以是空的,就像上面一样
            \item 可以把两个结合起来,就是一个空行了
            \item 内容一般都用列表(小标题)显示
            \item 最好不要有大段的话
        \end{itemize}
    \end{frame}

    \begin{frame}
        \frametitle{简单图片示例}
        \begin{itemize}
            \item 一般用center环境居中,图片宽度和高度都是可调的
            \item 这个是宽度为40\%页面宽度的
            \item 也可以设置height,下面就是首页的背景
        \end{itemize}
        \begin{center}
            \includegraphics[width=.4\paperwidth]{pic/bg1.png}
        \end{center}
    \end{frame}

    \section{(稍)复杂页面示例} 
    
    \begin{frame}
        \frametitle{嵌套列表示例}

        \begin{itemize}
            \item 这是第一层列表
            \item 这也是第一层列表
            \begin{itemize} 
                \item 这是第二层列表
                \item 这也是第二层列表
            \end{itemize}
            \item 这还是第一层列表
            \begin{itemize}
                \item 这是另一个第二层列表
                \item 这也是另一个第二层列表
            \end{itemize}
            \item 怎么还是第一层列表?
        \end{itemize}
    \end{frame}

    \begin{frame}
        \frametitle{分栏示例}
        \begin{columns}

            \column{.5\textwidth} 
            \begin{itemize}
                \item 这是第一栏,占行宽的50\%
                \item 通过.5参数实现的,也可以改成其他值,如.4
                \item 可以放列表,也可以放图片
                \item 就像右边的其它页背景一样,默认左对齐
                \item 可以像之前那样用center居中
            \end{itemize}

            \column{.4\textwidth}
                \includegraphics[height=0.4\textheight]{pic/bg2.png}

        \end{columns}
    \end{frame}

    \begin{frame}
        \frametitle{带动画效果的列表}
        \begin{itemize}
            \item<4-> 真的是自定义的,这里就藏了一个
            \item<1-> 这是第一个显示的
            \begin{itemize}
                \item<3-> 显示顺序同样适用于嵌套列表
            \end{itemize}
            \item<2-> 显示顺序完全是自定义的
        \end{itemize}
    \end{frame}
    
    \section{一个可能有用的HACK}
    
    \begin{frame}
        \frametitle{距离的人工调整}
        \vspace{-6cm} % 垂直距离的HACK
        \begin{itemize}
            \item 比如你觉得正文内容和标题离的太远
            \item 这样就可以减少了,vspace命令是垂直距离
            \item v 就是 vertical
            \item 参数值可以是正的(扩大),也可以是负的(缩小)
            \item 水平距离,就是hspace,语法也一样
            \item h 就是 horizontal
        \end{itemize}
    \end{frame}
    
    \section{参考文献}
    
    \begin{frame}
        \frametitle{参考文献}
        \vspace{-1.5cm}
        \begin{thebibliography}{}
            \bibitem[1]{1} 参考文献比较hack
            \bibitem[2]{2} 方括号里的序号就是前面显示的序号,所以不能乱
            \bibitem[3]{3} 大括号里的是引用标记,应该用不到,可以随便写
            \bibitem[4]{4} 条目内容的话,直接把论文里整理好的粘贴过来就行了
            \bibitem[5]{5} 如果有特殊符号,就要稍微注意处理一下了
        \end{thebibliography}
    \end{frame}

    % Slide主要内容的结束
    % 最后一页自动生成

    \begin{frame}
        \vspace{4cm}
    
        \bf \zihaopt{60}{
            \begin{tabularx}{\textwidth}{c}
                \makebox[\textwidth]{\hfill \color{blue}{敬请批评指正!} \hfill}
            \end{tabularx}
        }
        
        \vspace{1cm}
        
        \bf \zihaopt{24}{
            \setlength{\extrarowheight}{-3cm}
            \begin{tabularx}{\textwidth}{@{\hspace{6cm}}ccc}
                \makebox[4cm]{学\hfill 号} & : & \makebox[4cm]{\hfill \usebeamertemplate{sid} \hfill} \\
                \makebox[4cm]{姓\hfill 名} & : & \makebox[4cm]{\hfill \insertauthor \hfill} \\
            \end{tabularx}
        }
    \end{frame}
\end{document}
